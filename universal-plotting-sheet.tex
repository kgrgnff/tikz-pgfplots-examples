% Universal Plotting Sheet
% Author: Kolja Gerganoff

%\documentclass{minimal}
%\documentclass[margin=0pt]{standalone}
\documentclass[11pt]{article}
\usepackage{nopageno} %\thispagestyle{empty}

\usepackage{helvet}
\usepackage{mathptmx}

\usepackage[dvipsnames]{xcolor}

\usepackage{geometry}
\geometry{
	a4paper,
	landscape,
	total={257mm,170mm},
	left=27mm,
	top=21mm,
}
\usepackage{tikz} 
\usetikzlibrary{fpu,math}
\usepackage{siunitx}
\usepackage{verbatim}
\usepackage{ifthen}

\usepackage{pgfplots}
\pgfplotsset{compat=1.17}
\pgfplotsset{every axis/.append style={
	%axis x line=none,
	axis line style={-,color=OliveGreen},
	tick label style={font=\footnotesize} %\small \scriptsize \footnotesize \tiny \si{\degree}
}}

\newcommand\degfs{\fontsize{9pt}{11pt}\selectfont}

\begin{comment}
:Title: Universal Plotting Sheet
:Tags: Marine Sheets
:Author: Kolja Gerganoff
\end{comment}

\begin{document}

%------------------------------------
% Enter latitude and longitude of your position
%------------------------------------
\def\latitude{54.32325} 	% latitude in decimal format
\def\latns{N} % N or S
%\def\latns{S} 
\def\longitude{10.13224}	% longitude in decimal format
\def\longew{E} % E or W
%\def\longew{W}

% 1 -> draw lines of latitude, latitude must be given above
% 0 -> don't draw lines of latitude
\def\drawlatitude{1} 
%\def\drawlatitude{0} 

% Constants for the circle and lines of longitude
\def\maxradius{8.01cm}
\def\maxleft{12.0cm}
\def\maxright{12.0cm}
\def\radius{6.0cm}
\def\twentymindist{2.0cm}
\def\onedeglen{0.2cm}
\def\fivedeglen{0.28cm}
\def\tendeglen{0.32cm}
\def\labeldist{0.5cm}
\def\outerlabeldist{0.80cm}
\def\labelydist{0.6cm}
\def\axismargin{0.7cm}
\def\innerlinemargin{0.88cm}
\def\outerlinemargin{0.05cm}
\def\onedegrad{\radius - \onedeglen}
\def\fivedegrad{\radius - \fivedeglen}
\def\tendegrad{\radius - \tendeglen}
\def\labelrad{\radius - \labeldist}
% Constants for the reference field of the latitude length
\def\d{\radius / 30.0}
\def\xaxisoffset{\radius - 1cm}
\def\yaxisoffset{-\radius + 0.6cm}
\def\yscale{15.2}
\def\xaxisoffsetb {\xaxisoffset}
\def\yaxisoffsetb {\yaxisoffset}
\def\dy{1.0/\yscale}
\tikzmath{
	\sinlat = \radius * sin(\latitude)/1cm;
	\coslat = \radius * cos(\latitude)/1cm;
}

\begin{centering}
	\begin{tikzpicture}[scale=1.0]
		\begin{scope}

			%------------------------------------
			% Helper lines
			%------------------------------------
			\node[draw, very thin, white, circle, inner sep=3mm] (a) at (0,0) {};
			\foreach \x in {45, 135, ..., 360} \draw[very thin, gray!40] (a) -- (\x:\radius);
			% draw lines of longitude
			\ifnum \drawlatitude=1
				\foreach \x in {-2,-1,...,2}{
						\ifnum \x=0 \else
							\draw [thin, gray!40] (\x * \coslat, -\maxradius) -- (\x * \coslat, \maxradius);
						\fi
						\node at (\x * \coslat, \maxradius+\labeldist)
						{%
							\color{OliveGreen} \degfs
							\pgfmathparse{\x + \longitude - 0.5}
							\pgfmathprintnumber[
								fixed,
								precision=0
							]{\pgfmathresult}\si{\degree}\longew%
						};
					}
			\fi
			% draw lines of latitude
			\foreach \x in {-1,0,1}{
					\ifnum \x=0 \else
						\draw [thin, gray!40] (-\maxleft, \x * \radius) -- (\maxright, \x * \radius);
					\fi
					\ifnum \drawlatitude=1
						\node at (\maxright+\labeldist, \x * \radius)
						{%
							\color{OliveGreen} \degfs
							\pgfmathparse{\x + \latitude - 0.5}
							\pgfmathprintnumber[
								fixed,
								precision=0
							]{\pgfmathresult}\si{\degree}\latns%
						};
					\fi
				}

			%------------------------------------
			% Circle with ticks
			%------------------------------------
			\draw[thick, color=OliveGreen] (0,0) circle (\radius);

			% ticks every 1 degree if not mod10,mod5 
			\foreach \i
			[evaluate=\i as \x using {ifthenelse( int(mod(\i,5))==0 || int(mod(\i,10))==0, 0,1)}] in {0,...,359} {%
				\ifnum \x=1
					\draw[color=OliveGreen] (\i:\onedegrad) -- (\i:\radius);
				\fi
			}
			% ticks every 5 degrees
			\foreach \i
			[evaluate=\i as \x using {int(mod(\i,10))}] in {0,5,...,355} {%
				\ifnum \x=0 \else
					\draw[color=OliveGreen] (\i:\fivedegrad) -- (\i:\radius);
				\fi
			}
			% labels and longer ticks at every 10 degrees
			\foreach \x in {0,10,...,350} {%
				\ifthenelse{\x>90 \AND \x<270}
				{\node[scale=1.0, rotate=180-\x] at (360-\x+90:\labelrad){\color{OliveGreen} \degfs \x};}
				{\node[scale=1.0, rotate=\x*-1] at (360-\x+90:\labelrad) {\color{OliveGreen} \degfs \x};}
				\draw[thick, color=OliveGreen] (\x:\tendegrad) -- (\x:\radius);
			}
			% labels for latitude
			\foreach \x in {10,20,...,70} {%
				{\node[scale=1.0, rotate=-\x] at (360-\x:\labelrad+\outerlabeldist){\color{OliveGreen} \degfs \x\si{\degree}};}
				{\node[scale=1.0, rotate=\x] at (360+\x:\labelrad+\outerlabeldist){\color{OliveGreen} \degfs \x\si{\degree}};}
			}

			%------------------------------------
			% Vertical axis
			%------------------------------------
			\draw[thick, color=OliveGreen] (0,-\radius + \axismargin) -- (0,\radius - \axismargin);
			\draw[thick, color=OliveGreen] (0,\radius + \outerlinemargin) -- (0,\maxradius);
			\draw[thick, color=OliveGreen] (0,-\radius - \outerlinemargin) -- (0,-\maxradius);

			%------------------------------------
			% Horizontal axis and grid
			%------------------------------------
			% "zero" longitude line
			\draw[thick, color=OliveGreen] (-\radius - \outerlinemargin,0) -- (-\maxleft,0);
			\draw[thick, color=OliveGreen] (\radius + \outerlinemargin,0) -- (\maxright,0);
			\draw[thick, color=OliveGreen] (-\radius + \innerlinemargin,0) -- (\radius -  \innerlinemargin,0);

			% lines at 0deg+20' and -1deg40' (-20' from center line) 
			\draw[thin, gray!40] (-\outerlinemargin,\radius +  \twentymindist) -- (-\maxleft, \radius +  \twentymindist);
			\draw[thin, gray!40] (\innerlinemargin, \radius +  \twentymindist) -- (\maxright, \radius +  \twentymindist);
			\draw[thin, gray!40] (-\outerlinemargin, -\radius - \twentymindist) -- (-\maxleft, -\radius -  \twentymindist );
			\draw[thin, gray!40] (\innerlinemargin, -\radius - \twentymindist) -- (\maxright, -\radius - \twentymindist);

			% ticks every 1 degree if not mod10,mod5 within the circle
			\foreach \i
			[evaluate=\i as \x using {ifthenelse( int(mod(\i,5))==0 || int(mod(\i,10))==0, 0,1)}] in {-52,-51,...,52} {%
					\ifnum \x=1
						\draw[color=OliveGreen] (0,\i cm / 10) -- (\onedeglen,\i cm / 10);
					\fi
			}
			%  ticks every 1 degrees above the circle
			\foreach \i
			[evaluate=\i as \x using {ifthenelse( int(mod(\i,5))==0 || int(mod(\i,10))==0, 0,1)}] in {1,...,19} {%
					\ifnum \x=1
						\draw[color=OliveGreen] (0,\radius + \i cm / 10) -- (\onedeglen,\radius + \i cm / 10);
					\fi
			}
			%  ticks every 1 degrees below the circle
			\foreach \i
			[evaluate=\i as \x using {ifthenelse( int(mod(\i,5))==0 || int(mod(\i,10))==0, 0,1)}] in {1,...,19} {%
					\ifnum \x=1
						\draw[color=OliveGreen] (0,-\radius - \i cm / 10) -- (\onedeglen,-\radius - \i cm / 10);
					\fi
			}

			% ticks every 5 degrees within the circle
			\foreach \i
			[evaluate=\i as \x using {ifthenelse( int(mod(\i,5))==0 && int(mod(\i,10))!=0, 0,1)}] in {-52,-51,...,52} {%
					\ifnum \x=0
						\draw[color=OliveGreen] (0,\i cm / 10) -- (\fivedeglen,\i cm / 10);
					\fi
			}
			% ticks every 5 degrees above the circle
			\foreach \i
			[evaluate=\i as \x using {ifthenelse( int(mod(\i,5))==0 && int(mod(\i,10))!=0, 0,1)}] in {1,...,19} {%
					\ifnum \x=0
						\draw[color=OliveGreen] (0,\radius + \i cm / 10) -- (\fivedeglen,\radius + \i cm / 10);
					\fi
			}
			% ticks every 5 degrees below the circle
			\foreach \i
			[evaluate=\i as \x using {ifthenelse( int(mod(\i,5))==0 && int(mod(\i,10))!=0, 0,1)}] in {1,...,19} {%
					\ifnum \x=0
						\draw[color=OliveGreen] (0,-\radius - \i cm / 10) -- (\fivedeglen,-\radius - \i cm / 10);
					\fi
			}

			% ticks and labels every 10 degrees within the circle
			\foreach \i
			[evaluate=\i as \x using {ifthenelse(int(mod(\i,10))==0, 0,1)}] in {1,...,59} {%
			\ifnum \x=0
				\draw[thick, color=OliveGreen] (0,\i cm / 10) -- (\tendeglen,\i cm / 10);
				{\node[scale=1.0] at (\labelydist, \i cm / 10){\color{OliveGreen} \degfs \i´};}
				\draw[thick, color=OliveGreen] (0,-\radius + \i cm / 10) -- (\tendeglen,-\radius + \i cm / 10);
				{\node[scale=1.0] at (\labelydist,-\radius + \i cm / 10){\color{OliveGreen} \degfs \i´};}
			\fi
			}
			%  ticks and labels every 10 degrees above the circle
			\foreach \i
			[evaluate=\i as \x using {ifthenelse(int(mod(\i,10))==0, 0,1)}] in {1,...,20} {%
			\ifnum \x=0
				\draw[thick, color=OliveGreen] (0,\radius + \i cm / 10) -- (\tendeglen,\radius + \i cm / 10);
				{\node[scale=1.0] at (\labelydist, \radius + \i cm / 10){\color{OliveGreen} \degfs \i´};}
			\fi
			}
			% ticks and labels every 10 degrees below the circle
			\foreach \i
			[evaluate=\i as \x using {ifthenelse(int(mod(\i,10))==0, 0,1)}] in {59,...,40} {%
			\ifnum \x=0
				\draw[thick, color=OliveGreen] (0,-2*\radius + \i cm / 10) -- (\tendeglen,-2*\radius + \i cm / 10);
				{\node[scale=1.0] at (\labelydist,-2*\radius + \i cm / 10){\color{OliveGreen} \degfs \i´};}
			\fi
			}
		\end{scope}

		%------------------------------------
		% Reference field for determining the length of a longitudal degree
		%------------------------------------
		\begin{scope}[xshift=\xaxisoffset,yshift=\yaxisoffset]
			\begin{axis}[
					ymin=-0.08,ymax=70.08,
					xmin=-0.05,xmax=60.06,
					y = 0.8 * \radius / 70,
					x = \radius / 60,
					xtick = {0,10,...,60}, %\empty,´
					xticklabels = {60´,50´,40´,30´,20´,10´,0},% {0,10´,20´,30´,40´,50´,60´},
					ytick = {0,10,...,70},
					yticklabels = {0\si{\degree}, 10\si{\degree},20\si{\degree},30\si{\degree},40\si{\degree},50\si{\degree},60\si{\degree},70\si{\degree}},
					minor y tick num=9,
					minor x tick num=0,
					yticklabel pos=right,
					ytick pos=right,
					major tick length=0.15cm,
					minor tick length=0.05cm,
					axis y line = left,
					axis x line = bottom,
					axis y line=right,
					axis line style =  {-},
					axis y line shift=\radius/18.1,
					scale=1.0,
					color=OliveGreen,
					y tick label style={xshift=-.2em},
					x tick label style={yshift=.1em},
					tick style={color=OliveGreen},
					x tick style={color=white},
					y tick style={color=OliveGreen},
					thin
				]
				% vertical 6/6 degree, 5/6, ...  lines of longitude
				\foreach \i in {0.0, 1.0, 2.0, 3.0, 4.0, 5.0, 6.0}{
						\addplot[domain=0.0:70.0,samples=100,variable=\u](%
						{(\radius - ((\radius - \i cm) * cos(\u)))/1cm*10 },%
						{\u }%
						);
				}

				% 10 seconds of longitude lines
				\foreach \i in {\d, 2*\d, 3*\d, 4*\d}{
						\addplot[domain=0.0:70.0,samples=100,variable=\u](%
						{(\radius - ((\radius - 5.0 cm - \i) * cos(\u)))/1cm*10},%
						{(\u}%
						);
				}

				% horizontal lines
				\ifnum \drawlatitude=1
					\addplot[color=red,domain=(\radius + 0.1pt - \radius* cos(\latitude))/1cm*10: \radius/1cm*10,samples=100,variable=\u](%
					{\u},%
					{\latitude}%
					);
				\fi
				\addplot[line cap=rect,domain=(\radius + 0.1pt - \radius* cos(70.0))/1cm*10: \radius/1cm*10,samples=100,variable=\u](%
				{\u},%
				{70.0}%
				);
				\foreach \i in {60.0, 50.0, 40.0, 30.0, 20.0, 10.0}{%
						\addplot[domain=(\radius - \radius * cos(\i))/1cm*10: \radius/1cm*10,samples=100,variable=\u](%
						{\u},%
						{\i}%
						);
				}
			\end{axis}
		\end{scope}
	\end{tikzpicture}
\end{centering}

\end{document}


